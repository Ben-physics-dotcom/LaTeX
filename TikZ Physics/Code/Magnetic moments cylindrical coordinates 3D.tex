\documentclass{standalone}

\usepackage[siunitx,americanvoltages, europeanresistors,americancurrents]{circuitikz}
\usepackage{color}
\usepackage{graphicx}%Für Grafiken
\usepackage{rotating} % lässt Grafiken rotieren
\usepackage{mathtools}% mathematische Werkzeuge
\usepackage{amsmath}% Mathetools
\usepackage{amsfonts}% Mathetools
\usepackage{physics}
\usepackage{amssymb}% Symbole wie Natürliche Zahlen
\usepackage{geometry}
\usepackage{caption} % Unter-/Überschriften für Bilder, Grafiken und Tabellen
\usepackage{tikz}
\usepackage{amsthm}

% tikz libraries
\usetikzlibrary{arrows}
\usetikzlibrary{3d}
\usetikzlibrary{angles, quotes, shapes, decorations.markings, calc, arrows.meta}

% mathmatical commands
\newcommand{\AAA}{\mathbf{A}}
\newcommand{\R}{\mathbb{R}}
\newcommand{\N}{\mathbb{N}}
\newcommand{\p}{\mathcal{P}}
\newcommand{\I}{\infty}
\newcommand{\ve}{\varepsilon}
\newcommand{\vp}{\varphi}

\begin{document}

			\begin{tikzpicture}[>=stealth]
				\filldraw[draw=white,fill=white] (-1.9,-3) rectangle (3.25,1.9); % delete this line, if you use the grafic
				% coordinate system
				\draw[->,very thick] (-2,0) -- (2,0);
				\draw[->,very thick] (0,-2) -- (0,2);
				\draw[->,very thick] (0,0,-2) -- (0,0,2);
				
				% m1
				\filldraw[fill=blue,draw=blue] (0,0) circle (.05);
				\begin{scope}[canvas is xz plane at y=0.05]
					\draw[color=blue,<-] ({0.5*cos(30)},{0.5*sin(30)}) arc (30:350:.5);
				\end{scope}
				\draw[very thick] (0,.1) -- (0,1);
				
				% m2
				
				\begin{scope}[rotate around x=30]
					\filldraw[fill=blue,draw=blue] (1.5,1.5) circle (0.05);
					\draw[color=blue,->] ({0.5*cos(30)+1.5},{0.5*sin(30)+1.5}) arc (30:350:.5);
				\end{scope}
				
				% Winkel phi:
				\begin{scope}[canvas is xz plane at y=0.05]
					\draw[<-] ({cos(100)},{sin(100)}) arc (100:160:1);
				\end{scope}
				
				% text:
				\node[scale=.7,below left] at ({cos(120)},0,{sin(120)}) {$\vp$};
				\node[scale=.7,left] at (0,0,2) {r};
				\node[scale=.7,above] at (0,2) {z};
				\node[scale=.7,above 
				left,color=blue] at (0,0.05) {$ m_1 $};
				\node[scale=.7,below,color=blue] at (1.5,1.5) {$ m_2 $};
		\end{tikzpicture}
	
\end{document}
