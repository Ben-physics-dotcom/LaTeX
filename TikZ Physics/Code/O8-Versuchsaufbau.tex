\documentclass{standalone}

\usepackage{imakeidx}
\usepackage{ragged2e}
\usepackage{setspace} % Um den Zeilenabstand zu ändern.
\usepackage{gensymb}
%\usepackage[siunitx,americanvoltages, europeanresistors,americancurrents]{circuitikz}
\usepackage{color}
\usepackage{wasysym}
\usepackage{array} % Verwendung von Matrizen
\usepackage{booktabs} % Schöne Tabellen beziehungsweise sie sehen damit professioneller aus.
\usepackage{tabulary} % Ähnlich wie tabularx, ermöglicht aber das ändern der Ausrichtung der Spalten.
\usepackage{tabularx} % Tabellen mit automatischen Zeilenumbruch.
\usepackage{enumitem}
\usepackage[T1]{fontenc}% fontec und inputenc ermöglichen
\usepackage{graphicx}%Für Grafiken
\usepackage{rotating} % lässt Grafiken rotieren
\usepackage{mathtools}% mathematische Werkzeuge
\usepackage{amsmath}% Mathetools
\usepackage{amsfonts}% Mathetools
\usepackage{physics}
\usepackage{amssymb}% Symbole wie Natürliche Zahlen
\usepackage{geometry}
%\usepackage{bibtex} 
\usepackage{tablefootnote}% Fußnoten in Tabellen
\usepackage{float}% für eingebundene Bilder
\usepackage{fancyhdr} % Seiten schöner gestalten, insbesondere Kopf- und Fußzeile
\usepackage{ulem} 
\usepackage[ngerman]{babel} % Worttrennung nach der neuen Rechtschreibung und deutsche Bezeichnungen. babelfunktion wird wegen Literatur gebraucht.
\usepackage{subfloat} % Was macht diese Packet?
\usepackage{caption} % Unter-/Überschriften für Bilder, Grafiken und Tabellen
\usepackage{subcaption}
\usepackage{titling}% Titel
%\usepackage[style=alphabetic,firstinits=true]{biblatex} %biblatex mit alphabetic laden. alphbetic=Zitationsstil
\usepackage{bookmark}
%\usepackage[printonlyused]{acronym}
\usepackage{tikz}
\usepackage{url}
\usepackage{amsthm}
%\usepackage{subfigure}
\usepackage{pdfpages}
\usepackage{abstract}
\usepackage{hyperref} % In die ersten Klammern kommt der Link, in die zweite Klammer der Titel/Angaben zum Link.
%\makeindex[title=Stichwortverzeichnis,options=-s Index-Formatierung.ist,intoc]
%\addbibresource{Blog-und-Kurse.bib}

\usetikzlibrary{arrows}
\usetikzlibrary{3d}
\hypersetup{
	colorlinks=true,
	linkcolor=blue,
	filecolor=magenta,      
	urlcolor=cyan,
}


% tikz library
\usetikzlibrary{angles, quotes, shapes, decorations.markings, calc, arrows.meta}




\newcommand{\AAA}{\mathbf{A}}
\newcommand{\R}{\mathbb{R}}
\newcommand{\N}{\mathbb{N}}
\newcommand{\p}{\mathcal{P}}
\newcommand{\sq}{^{2}}
\newcommand{\I}{\infty}
\newcommand{\ve}{\varepsilon}
\newcommand{\vp}{\varphi}
\begin{document}
%	\begin{figure}[ht!]
%		\centering
		\begin{tikzpicture}[>=stealth]
			\draw[white] (0,0) rectangle (8,-2);
			\draw (-7,0) -- (7,0);
			\draw (-6.0,0) -- (-6.0,1);
			\draw (-3.5,0) -- (-3.5,1.1);
			\draw (3,0) -- (3,.5);
			
			% Laser
			\filldraw[fill=white,draw=black] (-7,1) rectangle (-5,1.8);
			
			% Lochblende
			\filldraw[fill=white,draw=black] (-3.5,1.4) circle (.3); 
			\filldraw[fill=white,draw=black] (-3.5,1.4) circle (.05); 
			
			%Pfeil
			\draw[very thin,<->] (-3.5,1.) -- (2.45,1.);
			\node[scale=.7,below] at (-.5,1) {d\label{IV,d}};
			
			% Fotodetektor
			\filldraw[fill=white,draw=black] (2.25,.5) rectangle (3.75,.8);
			\filldraw[fill=white,draw=black] (3.75,.9) rectangle (3.85,.4);
			\foreach \x in {0,.3,...,1.5}
			\draw ({2.3+\x},.65) -- ({2.3+\x},.8);
			\foreach \x in {0,.3,...,1.5}
			\draw ({2.45+\x},.7) -- ({2.45+\x},.8);
			\foreach \x in {0,.0375,...,1.425}
			\draw ({2.3+\x},.75) -- ({2.3+\x},.8);
			\draw (2.45,.8) -- (2.45,1.1);
			\draw (2.45,1.4) circle (.3);
			
			% Ampere-Meter
			\draw[thin] ({2.45+cos(315)*.3},{1.4+sin(315)*.3}) -- ({5.4+cos(315)*.3},{1.4+sin(315)*.3});
			\draw (5.8,1.4) circle (.3);
			\node[scale=.8] at (5.8,1.4) {A\label{Amperemeter}};
			
			% Laserstrahl
			\draw[red] (-5,1.4) -- (-3.45,1.4);
			
			\foreach \r in {.03,.05,.09,.11,.14,.16} 
			\draw[color=red!60!white] (2.4,1.4) circle (\r);
			\draw[red] (-3.2,1.4) -- (2.4,1.4);
			
			% Beschriftung
			\node[scale=.7,above] at (-6,1.4) {L\label{IV,L}};
			\node[scale=.7,below] at (-6,1.4) {$\lambda$\label{IV,lambda}};
			\node[scale=.7,above] at (-3.5,1.7) {B\label{IV,B}};
			\node[scale=.7,above] at (2.45,1.7) {F\label{IV,F}};
			\node[scale=.7,right] at (3.85,.65) {M\label{IV,M}};
		\end{tikzpicture}
%		\caption{Versuchsaufbau zur relativen Intensitätsverteilung\index{Intensität}\index{Intensität!Verteilung!relativ}  }
%		\label{Abb: Versuchsaufbau rel. I.vert.}
%	\end{figure}
	
	
%	\noindent	Wir haben einen Helium-Neon-Laser\index{Laser}\index{Laser!Helium-Neon} \hyperref[IV,L]{L}, der einen Lichtstrahl
%	\index{Licht}\index{Licht!Strahl}
%	mit einer Wellenlänge\index{Wellenlänge} \hyperref[IV,lambda]{$\lambda$} von 633nm emittiert. Die Lochblende\index{Blende}\index{Blende!Loch-} \hyperref[IV,B]{B} erzeugt ein Ringmuster\index{Ring}\index{Ring!Muster} auf die Fotodiode\index{Fotodiode} \hyperref[IV,F]{F}, welche als Detektor\index{Detektor} dient. Die Fotodiode erzeugt vom Laserstrahl\index{Laser!Strahlt} mittels photoelektrischen Effekt\index{photoelektrischer Effekt} einen Strom\index{Strom}, der mit Hilfe eines Amperemeters\index{Amperemeter} \hyperref[Amperemeter]{A} gemessen werden kann. Um die relative Intensitätsverteilung\index{Intensität}\index{Intensität!Verteilung}\index{Intensität!Verteilung!relativ} zu messen, kann der Detektor\index{Detektor} um ein hundertstel Millimeter mittels einer Millimeter-Schiene\index{Millimeter-Schiene} \hyperref[IV,M]{M} verschoben werden und somit die relative Intensität\index{Intensität} der Ringe\index{Ring} gemessen werden.
\end{document}
